\section{MacWilliams' Identity}

The following proofs follows from the paper ``Weight Enumeration and the Geometry of Linear Codes'' \cite{Greene} by C. Greene.

\subsection{Corank-Nullity Generating Function of the Dual Code} \label{app:Whitney}
This is an alternate proof to \autoref{lem:w7_corank_nullity}.

\begin{proof}
    Denote $E=\{1,2,\ldots,n\}$ be the indices and $S$ a subset of $E$. We denote $r(S)$ the rank of the columns $S$ of a generator matrix of a code $\mathcal{C}$, and $r^*(S)$ the rank of the columns $S$ of a generator matrix of the dual code $\mathcal{C}^\perp$.

    A result from \textit{matroid theory} states that
    \begin{equation}
        r^*(S) = \abs{S} + r(E\backslash S) - r(E).
    \end{equation}
    This equation is related to Tutte-Grothedieck invariance, and is in essence, derived from combinatorics arguments. Hence, this proof actually still uses counting. Nevertheless, let us assume that this equation is known and continue on.

    Let us consider the corank-nullity generating function of the dual code:
    \begin{align*}
        W_{\mathcal{C}^\perp}(x,y) &= \sum_{S\subseteq E} x^{r^*(G) - r^*(S)} y^{\abs{S} - r^*(S)} \\
        &= \sum_{S\subseteq E} x^{(n-k) - (\abs{S} + r(E\backslash S) - r(G))} y^{\abs{S} - (\abs{S} + r(E\backslash S) - r(G))} \\
        &= \sum_{S\subseteq E} y^{r(G) - r(E\backslash S)} x^{(n-\abs{S}) - r(E\backslash S)} = W_\mathcal{C}(y,x).
    \end{align*}
    Thus it is shown.
\end{proof}


\subsection{Greene's Lemma} \label{app:Greene}
This is an alternate proof to \autoref{lem:w7_Greene}.

\begin{proof}
    Let $E=\{1,2,\ldots,n\}$ be the indices for a codeword. The support of a codeword $c\in\mathcal{C}$ is defined as
    \begin{equation*}
        \mathrm{supp}(c) \defeq \setdef{i\in E}{c_i=1}.
    \end{equation*}
    The weight enumerator can then be rewritten as
    \begin{equation*}
        A_\mathcal{C}(x,y) = \sum_{S\subseteq E} \#\setdef{c\in\mathcal{C}}{\mathrm{supp}(c)=S} x^{\abs{S}} y^{n-\abs{S}}.
    \end{equation*}
    The RHS of the statement can then be reduced to
    \begin{align*}
        (y-x)^kx^{n-k} W_\mathcal{C}\left(\frac{2x}{y-x},\frac{y-x}{x}\right) &= \sum_{T\subseteq E} 2^{k-r(T)} x^{n-\abs{T}} (y-x)^{\abs{T}} \\
        &= \sum_{T\subseteq E}\sum_{m=0}^{\abs{T}} (-1)^{\abs{T}-m} 2^{k-r(T)} \binom{\abs{T}}{m} x^{n-m} y^m \\
        &= \sum_{T\subseteq E} \sum_{S\subseteq T} (-1)^{\abs{T}-\abs{S}} 2^{k-r(T)} x^{n-\abs{S}}y^{\abs{S}} \\
        &= \sum_{S\subseteq E} \sum_{T\supseteq S} (-1)^{\abs{T}-\abs{S}} 2^{k-r(T)} x^{n-\abs{S}}y^{\abs{S}}.
    \end{align*}
    Next, we replace $S\mapsto S^\mathrm{c}$ and $T\mapsto T^\mathrm{c}$,
    \begin{align*}
        &= \sum_{S^\mathrm{c}\subseteq E} \sum_{T^\mathrm{c}\supseteq S^\mathrm{c}} (-1)^{\abs{T^\mathrm{c}}-\abs{S^\mathrm{c}}} 2^{k-r(T^\mathrm{c})} x^{n-\abs{S^\mathrm{c}}}y^{\abs{S^\mathrm{c}}} \\
        &= \sum_{S\subseteq E} \left(\sum_{T\subseteq S} (-1)^{\abs{S}-\abs{T}} 2^{k-r(T^\mathrm{c})}\right) x^{\abs{S}} y^{n-\abs{S}}.
    \end{align*}
    We need to show that the term in the bracket is equal to $\#\setdef{c\in\mathcal{C}}{\mathrm{supp}(c)=S}$. This can easily be done by introducing the following lemma.
    \begin{lemma}[M{\"o}bius Inversion]
        For set functions $f$ and $g$, we have
        \begin{equation}
            f(S) = \sum_{T\subseteq S} (-1)^{\abs{S}-\abs{T}} g(T) \Leftrightarrow g(S) = \sum_{T\subseteq S} f(T).
        \end{equation}
    \end{lemma}
    \textit{Proof. (M\"obius inversion)}\;
        The lemma is simply proven below:
        \begin{align*}
            \sum_{T\subseteq S} (-1)^{\abs{S}-\abs{T}} \sum_{V\subseteq T} f(V) &= \sum_{V\subseteq S} \sum_{T\supseteq V} (-1)^{\abs{S}-\abs{T}} f(V) \\
            &= \sum_{V\subseteq S} (-1)^{\abs{S}-\abs{V}} f(V) \sum_{m=0}^{\abs{S}-\abs{V}} \binom{\abs{S}-\abs{V}}{m} (-1)^m \\
            &= \begin{dcases}
                \sum_{V\subseteq S} (-1)^{\abs{S}-\abs{V}} f(V) (1-1)^{\abs{S}-\abs{V}} = 0, &S\neq V;\\
                \sum_{V = S} (-1)^{\abs{S}-\abs{V}} f(V) = f(S), &S=V.
            \end{dcases}\;\;\;\;\;\square
        \end{align*}
        
    % \end{proof}
    With the lemma proven, let us continue with the proof of Greene's lemma. Let the set functions be $f(S) = \#\setdef{c\in\mathcal{C}}{\mathrm{supp}(c)=S}$ and $g(S) = 2^{k-r(S^\mathrm{c})}$. Since
    \begin{align*}
        \sum_{T\subseteq S} \#\setdef{c\in\mathcal{C}}{\mathrm{supp}(c)=T} &= \#\setdef{c\in\mathcal{C}}{\mathrm{supp}(c)\subseteq S} = 2^{k-r(S^\mathrm{c})},
    \end{align*}
    where the last equality is obtained since the generator matrix can be written in the full rank form below through column operations:
    \begin{equation*}
        G = \left[\begin{array}{c:c}
            \mathbb{F}_2^{r(S^\mathrm{c})\times\abs{S}} & \mathbb{F}_2^{r(S^\mathrm{c})\times(n-\abs{S})} \vspace{0.1cm} \\ \hdashline 
            \mathbb{F}_2^{r(S)\times\abs{S}} & 0
        \end{array}\right] \in\mathbb{F}_2^{k\times n}.
    \end{equation*}
    The Greene's lemma is finally proven by applying M\"obius inversion.
\end{proof}

\subsection{MacWilliam's Identity}  \label{app:MacWilliams}
This is an alternate proof to \autoref{thm:w6_McWilliams}.
\begin{proof}
    By directing applying the two lemmas above:
    \begin{align*}
        \frac{1}{\abs{\mathcal{C}}}A_{\mathcal{C}}(y-x,y+x) &= \frac{1}{2^k} (2x)^k(y-x)^{n-k} W_\mathcal{C}\left(\frac{y-x}{x},\frac{2x}{y-x}\right) \\
        &= x^k (y-x)^{n-k} W_{\mathcal{C}^\perp}\left(\frac{2x}{y-x},\frac{y-x}{x}\right) \\
        &= A_{\mathcal{C}^\perp}(x,y).
    \end{align*}
\end{proof}

