\section{What is the Course About?}
\begin{enumerate}
    \item Introduce polar codes \& LDPC codes.
    \item Analyze the performance of the two codes with probabilistic methods.
    \begin{enumerate}[label=(\roman*)]
        \item Martingale: used in the analysis of polar code.
        \item Approximate Message passing: can be seen as a generalization to belief propagation, which is often used in the calculations of statistical physics.
    \end{enumerate}
\end{enumerate}

\section{Grading}
\begin{itemize}
    \item Midterm exam: $30\%$
    \item Final exam or project*: $30\%$
    \item Homework: $40\%$
\end{itemize}
*Whether we want a final report in place of the exam is to be decided. The report will be to report a paper from conferences.

Further, attendance can be counted as 0.5 pts per signature, and it is only used to increase your score lest you failed.

The homework should be turned-in in PDF files on NTUCOOL, and should be treated as \LaTeX~practices. Recommended \LaTeX~compilers include Overleaf or VScode.

\newpage

\section{Topics and Schedules}
Weekly plan (each bullet point is one week)
\begin{enumerate}
    \item Noisy-channel coding review
    \begin{itemize}
        \item Channel capacity
        \item Finite block length tradeoff - Error exponent
        \item Finite block length tradeoff - Scaling exponent
    \end{itemize}
    \item Polar Codes
    \begin{itemize}
        \item Ar{\i}kan's polar codes
        \item introduction to Martingale
        \item List decoding
        \item Scaling exponent and error exponent of polar codes
        \item Polar codes with larger matrices
        \item (optional) Polar codes with larger alphabet
        \item (optional) quantum polar codes
    \end{itemize}
    \item LDPC codes
    \begin{itemize}
        \item Belief Propagation
        \item Introduction to Statistical mechanics
        \item Approximate Message Passing
        \item Spatially-coupled LDPC codes
        \item Probabilistic group testing
        \item (optional) quantum LDPC codes
    \end{itemize}
    \item Turbo codes (optional)
\end{enumerate}

\newpage


\section{Style Guide}
This is a reminder of the writing guidelines for the formatting of this lecture note.

\begin{itemize}
    \item \textbf{Bullet 1}: List titles should capitalize.
    \item \textbf{Bullet 2}: The first letter is capitalized if treated as a complete sentence.
\end{itemize}

A couple of other mathematical environments are also commonly used:

\begin{definition}[DEFINITION NAME] \label{def:w0_def}
This is a definition.
\end{definition} 

\begin{theorem}[THEOREM NAME] \label{thm:w0_thm}
This is a theorem.
\begin{equation}
    f(x) = x^2. \label{eq:w0_eq}
\end{equation}
\end{theorem}
\begin{proof}
    Derivation.
\end{proof}

\begin{lemma}[LEMMA NAME] \label{lem:w0_lem}
This is a lemma
\end{lemma}

\begin{remark}[Remark NAME] \label{rmk:w0_rmk}
This is a remark.
\end{remark}

One can reference using \autoref{def:w0_def}, \autoref{eq:w0_eq}, \autoref{thm:w0_thm}, \autoref{lem:w0_lem}, and \autoref{rmk:w0_rmk}.

\begin{paper}
    referencing a paper
\end{paper}

Lastly, unfinished texts can be \hl{highlighted}, important texts can be \textit{italized}.

\vspace{1cm}
\hrule
\begin{summary}{0}
    At the end of each week's lecture, a summary to the material is also given.
\end{summary}
