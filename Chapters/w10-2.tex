%% Section abstract
This chapter introduces us to plenty topics not directly related to coding theory. However, these topics such as group testing more or less utilize some results of coding theory (mostly Reed--Solomon code) and are interesting enough to be mentioned. The more you know, the better.


%% Start from here.
\section{Entropy Complexity}
\lecture{22 Apr.}

This section considers the following task: given $\mathrm{Ber}(1/2)$, how do we generate $\mathrm{Ber}(1/3)$? But before any of that, why is the generation of a random variable so important? If the random numbers generated for security purposes are not ``random'' enough, attackers knowing this knowledge can tailor their attack design to make the encryption scheme especially vulnerable. The generation of a random number is so vital in many of our algorithm design that our societies cannot afford it breaking.

A solution algorithm to the initial problem is shown below:

\begin{figure}[H]
    \centering
    \begin{tikzcd}
	& {\text{ root}} \\
	A && \bullet \\
	& B && {\text{redo}}
	\arrow["{1/2}"', from=1-2, to=2-1]
	\arrow["{1/2}", from=1-2, to=2-3]
	\arrow["{1/2}"', from=2-3, to=3-2]
	\arrow["{1/2}", from=2-3, to=3-4]
    \end{tikzcd}
    \caption{Generating $\mathrm{Ber}(1/3)$ from $\mathrm{Ber}(1/2)$.}
    \label{fig:w10_ber(1/3)}
\end{figure}

See the decision tree above in \autoref{fig:w10_ber(1/3)}. The algorithm starts at the root, and go to either left or right based on the output of a single $\mathrm{Ber}(1/2)$ trial. If ``redo'' is reached, one returns back to the root. The process is repeated until one lands on either $A$ or $B$. The probability of obtaining the two results is
\begin{equation*}
    \mathrm{Pr}\{A\}:\mathrm{Pr}\{B\} = \frac{1}{2}:\frac{1}{4}=2:1 \Rightarrow \mathrm{Pr}\{A\} = \frac{2}{3}, \mathrm{Pr}\{B\}=\frac{1}{3}.
\end{equation*}

This algorithm can be easily generalized. Suppose that we want to generate a random variable following $\mathrm{Pr}\{A\}:\mathrm{Pr}\{B\}:\mathrm{Pr}\{C\} = 1:2:3$, the algorithm can be drawn by a tree of height 3 (since $2^3 > 6 = 1 + 2 + 3$), as shown in the figure below:

\begin{figure}[H]
    \centering
    \begin{tikzcd}
	&&&& {\text{root}} \\
	&& \bullet &&&& \bullet \\
	& \bullet && \bullet & {} & \bullet && \bullet \\
	B & B & A & C && C & C & {\text{redo}} & {\text{redo}}
	\arrow[from=1-5, to=2-3]
	\arrow[from=1-5, to=2-7]
	\arrow[from=2-3, to=3-2]
	\arrow[from=2-3, to=3-4]
	\arrow[from=2-7, to=3-6]
	\arrow[from=2-7, to=3-8]
	\arrow[from=3-2, to=4-1]
	\arrow[from=3-2, to=4-2]
	\arrow[from=3-4, to=4-3]
	\arrow[from=3-4, to=4-4]
	\arrow[from=3-6, to=4-6]
	\arrow[from=3-6, to=4-7]
	\arrow[from=3-8, to=4-8]
	\arrow[from=3-8, to=4-9]
    \end{tikzcd}
    \caption{Generating $\mathrm{Pr}\{A\}:\mathrm{Pr}\{B\}:\mathrm{Pr}\{C\} = 1:2:3$ from $\mathrm{Ber}(1/2)$.}
    \label{fig:w10_tree_1}
\end{figure}

The figure above is a waster of resource, however, and we don't really need to do $\mathrm{Ber}(1/2)$ as many times as it suggests. Of course! We can reduce the tree above to a simpler one below:
\begin{figure}[H]
    \centering
    \begin{tikzcd}
	&&& {\text{root}} \\
	& \bullet &&&& \bullet \\
	B && \bullet & {} & C && {\text{redo}} \\
	& A && C
	\arrow[from=1-4, to=2-2]
	\arrow[from=1-4, to=2-6]
	\arrow[from=2-2, to=3-1]
	\arrow[from=2-2, to=3-3]
	\arrow[from=2-6, to=3-5]
	\arrow[from=2-6, to=3-7]
	\arrow[from=3-3, to=4-2]
	\arrow[from=3-3, to=4-4]
    \end{tikzcd}
    \caption{Reduced tree of \autoref{fig:w10_tree_1}.}
\end{figure}

Let the average number of trials in the above algorithm be $T$, it satisfies
\begin{equation*}
    T = 2\cdot \frac{1}{2} + 3\cdot \frac{1}{4} + 
    T\cdot\frac{1}{4} \Rightarrow T= \frac{7}{3}.
\end{equation*}
This is the optimal algorithm for this example, but in general, how do we know if we can do better? Is there a theoretical bound?

\begin{theorem}[(Knuth \& Yao, 1976) \cite{Complexity_Nonunif_Rand}]
    The amount of bits needed to describe a discrete random variable $X$ is on average at most
    \begin{equation}
        H(X) + 2 \text{ bits}.
    \end{equation}
\end{theorem}
This describes the \textit{entropy complexity} of a random variable. To demonstrate, consider the following example:
\begin{example}
    The worst case possible is if the distribution is the extreme case where the denominator of the distribution is $2^{k}+1$, then we need a tree of $k+1$ layers. Say $\mathrm{Pr}\{A\}:\mathrm{Pr}\{B\} = 2^k:1$, then expected number of $\mathrm{Ber}(1/2)$ trials will be
    \begin{align*}
        x &= \frac{1}{2} + \frac{1}{2^2}(x+1)+\frac{1}{2^3}(x+2)+\cdots+\frac{1}{2^{k+1}}(x+k)+\frac{k+1}{2^{k+1}} \\
        &= \frac{1}{2}\left(1+\frac{k+1}{2^k}\right) + \frac{1}{2}\left(2-\frac{k}{2^k} - \frac{1}{2^{k-1}}\right) + x\left(1-\frac{1}{2^{k}}\right) \\
        \rightarrow x &= \frac{2-(k+2)/2^k}{1-1/2^k}.
    \end{align*}
    The entropy is
    \begin{equation*}
        H(X) = \frac{2^k}{2^k+1}\lg\left(\frac{2^k+1}{2^k}\right) + \frac{1}{2^k+1}\lg(2^k+1).
    \end{equation*}
    We have $x-H(X)$ strictly increasing, and its limit is
    \begin{equation}
        x - H(X) \xrightarrow{k\rightarrow\infty} 2 \text{ bits}. 
    \end{equation}
\end{example}

It seems like $H(X)+2$ is the best guarantee that we can have, and the extra cost of 2 bits is inevitable. Or is it? Given a random vector $Y = (X_1,X_2)$, where $X_1$ and $X_2$ are both i.i.d. copies of $X$. Then, by the theorem above, we will only need
\begin{equation*}
    H(Y) + 2 = 2\left(H(X)+1\right) \text{ bits}.
\end{equation*}
On average, we only need $H(X)+1$ times $\mathrm{Ber}(1/2)$ to generate each $X$. If we want to generate $n$ copies of $X$, then the amount of trials needed becomes $H(X)+\frac{2}{n}$. In the limit, we can achieve the entropy. For example, to simulate rolling a dice, you would need $7+1/3$ times $\mathrm{Ber}(1/2$ trials; however, if you were to simulate two rolls of a dice, you only need to do the Bernoulli trial $13 + 1/3$ times \cite{Complexity_Nonunif_Rand}.

Though we saved the number of Bernoulli trials (the entropy complexity), it tradeoffs with a higher computational complexity.

\begin{remark}
    In the 1976 work by Knuth and Yao \cite{Complexity_Nonunif_Rand}, they discussed the optimality of generating any finite or countably infinite set of rational or irrational probabilities. It is definitely worth a read if one wishes to seek deeper understanding of the topic.
\end{remark}