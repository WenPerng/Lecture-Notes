\section{Algebra of Finite Fields}
\lecture{14 Apr.}
\lecture{Birthday}

Reed--Solomon codes are really useful and we would like to extend it from $\mathbb{F}_p$ to more general fields of order (size) $q=p^d$. Henceforth, some algebra is needed as prior knowledge. For more details, please refer to the book ``Essential Coding Theory'' \cite{Essential_Coding_Theory}.

As one already knows, a field $\mathbb{F}$ is a set with the usual addition, subtraction, multiplication, division, zero element, and multiplicative identity. A finite field is a field with finite elements. Finite fields are often called Galois fields, denoted by $\mathrm{GF}(q)$. Some common examples of a field are the usual $\mathbb{R}$, $\mathbb{C}$, and $\mathbb{F}_p \cong \mathbb{Z}_p$ where $p$ is a prime.

\begin{definition}[Character]
    The character of a field is defined as
    \begin{equation}
        \mathrm{char}\,\mathbb{F}\defeq \text{the smallest positive integer $c$ such that }\underbrace{1+1+\cdots+1}_{c\text{ copies}}=0.
    \end{equation}
\end{definition}
\begin{example}
    \begin{itemize}
        \item We have $\mathrm{char}\,\mathbb{C} = \infty$ and $\mathrm{char}\,\mathbb{R}=\infty$, they are termed \textit{characteristic-zero fields}.
        \item $\mathrm{char}\,\mathrm{GF}(p)=p$.
    \end{itemize}
\end{example}

\begin{theorem}
    For a finite field $\mathbb{F}$, $\mathrm{char}\,\mathbb{F}$ is a prime.
\end{theorem}
\begin{proof}
    If we have $\mathrm{char}\,\mathbb{F}=ab$ as a product of two integers $a$ and $b$ bigger than $1$, then
    \begin{equation*}
        0 = \underbrace{1+1+\cdots+1}_{\times ab} = \underbrace{(1+\cdots+1)}_{\times a} \cdot \underbrace{(1+\cdots+1)}_{\times b},
    \end{equation*}
    meaning that one of $a$ and $b$ cannot have their inverse defined: $1\cdot b = a^{-1} a b = a^{-1}0 = 0$. Thus, a contradiction is met, $\mathrm{char}\,\mathbb{F}$ can only be zero or prime.
\end{proof}

\begin{definition}[Base Field]
    For any $\mathbb{F}$, if $\mathrm{char}\,\mathbb{F}>0$, the field $\mathrm{GF(\mathrm{char}\,\mathbb{F})}$ is termed the \textit{base field}, or the \textit{prime subfield} of $\mathbb{F}$.
\end{definition}
If we treat $\mathbb{F}$ as a vector space, then its base field is like the field the vector space is defined on. See the theorem below for a complete statement. For more details, refer to \autoref{thm:w9_gen_fin_field}.

\begin{theorem}
    For any field $\mathbb{F}$, we have its order (size) being $\abs{\mathbb{F}} = p^n$, where $\mathrm{char}\,\mathbb{F}=p$ and $n$ is the dimension of $\mathbb{F}$ over its base field $\mathrm{GF}(p)$. We have $\mathbb{F}\cong\mathrm{GF}(p)^n$ as a vector space.
\end{theorem}

\begin{definition}
    The set $\mathrm{GF}(p)[x]$ is the polynomials in $x$ with coefficients in $\mathrm{GF}(p)$. We say that $f\in\mathrm{GF}(p)[x]$, or that $f$ is a polynomial over $\mathrm{GF}(p)$.
\end{definition}

\begin{definition}[Reducibility]
    A polynomial $f\in\mathrm{GF}(p)[x]$ is \textit{reducible} if there exists polynomials $g$, $h\in\mathrm{GF}(p)[x]$ with $\deg g$, $\deg h>0$, such that $f=gh$. Else, the polynomial is called \textit{irreducible}.
\end{definition}
\begin{example}
    \begin{itemize}
        \item The polynomial $x^2+1$ over $\mathrm{GF}(2)$ is reducible, since $(x+1)^2 = x^2+2x+1=x^2+1$.
        \item The polynomial $x^2+x+1$ over $\mathrm{GF}(2)$ can be shown to be irreducible by enumerating all possibilities.
    \end{itemize}
\end{example}

\subsection{Constructing a Finite Field}
It is interesting to note that every time when we have an irreducible polynomial over a field, we can construct another finite field! Our end goal is to construct, from $\mathrm{GF}(p)$ where $p$ is prime, into a general field $\mathrm{GF}(q)$ where $q=p^n$, showing its existence and uniqueness. The construction procedure is as below

\begin{definition}[Companion Matrix]
    Given a polynomial
    \begin{equation}
        f(x) = a_0 + a_1x + a_2x^2 + \cdots + a_{d-1}x^{d-1} + x^d \in \mathrm{GF}(p)[x],
    \end{equation}
    the companion matrix of $f$ is defined as
    \begin{equation}
        A = \left[\begin{matrix}
            0 & 0 & & & -a_0 \\
            1 & 0 & & & -a_1 \\
            0 & 1 & \ddots & & \vdots \\
            & & \ddots & 0 & -a_{d-2} \\
            & & & 1 & -a_{d-1}
        \end{matrix}\right] \in \mathrm{GF}(p)^{d\times d},
    \end{equation}
    with the right-most column containing the negative coefficients to $f$, the lower-left triangular containing an identity matrix of size $(d-1)\times(d-1)$, and the remaining entries all being 0.
\end{definition}

\begin{theorem}
    Let the companion matrix of the polynomial $f$ be denoted by $A$. The characteristic polynomial of $A$ is equal to $f$.
\end{theorem}
\begin{proof}
    This can be directly shown by computing the characteristic polynomial from definition: $\det(\lambda\mathbbm{1}-A)$. But we will show another method here. Observe the powers to $A$:
    \begin{align*}
        &A^0 = \left[\begin{matrix}
            1 &   &        &   &   \\
              & 1 &        &   &   \\
              &   & \ddots &   &   \\
              &   &        & 1 &   \\
              &   &        &   & 1
        \end{matrix}\right],
        A^1 = \left[\begin{matrix}
            0 &   &        &   & -a_0     \\
            1 & 0 &        &   & -a_1     \\
              & 1 & \ddots &   & \vdots  \\
              &   & \ddots & 0 & -a_{d-2} \\
              &   &        & 1 & -a_{d-1}
        \end{matrix}\right],
        A^2 = \left[\begin{matrix}
            0 &   &        &   & -a_0     & * \\
            0 & 0 &        &   & -a_1     & * \\
            1 & 0 & \ddots & 0 & \vdots  & * \\
              & 1 & \ddots & 0 & -a_{d-2} & * \\
              &   &        & 1 & -a_{d-1} & *
        \end{matrix}\right], \\
        &\ldots, A^{d-1} = \left[\begin{matrix}
            0      & -a_0     & * & \cdots & * \\
            0      & -a_1     & * & \cdots & * \\
            \vdots & \vdots  & * & \cdots & * \\
            0      & -a_{d-2} & * & \cdots & * \\
            1      & -a_{d-1} & * & \cdots & *
        \end{matrix}\right],
        A^d = \left[\begin{matrix}
            -a_0      & * & \cdots & * & * \\
            -a_1      & * & \cdots & * & * \\
            \vdots   & * & \cdots & * & * \\
            -a_{d-2}  & * & \cdots & * & * \\
            -a_{d-1}  & * & \cdots & * & * 
        \end{matrix}\right].
    \end{align*}
    From the Cayley--Hamilton theorem, we know that the characteristic polynomial, and that $A^{d}$ can be represented as a linear combination of $A^0$ to $A^{d-1}$. Comparing the coefficients to the first column, we can immediately see that
    \begin{equation*}
        A^{d} = -a_0 A^0 - a_1 A^1 - a_2 A^2 - \cdots - a_{d-1} A^{d-1}.
    \end{equation*}
    Hence it is shown.
\end{proof}

\begin{remark}
    In MATLAB and many other mathematical programs, given a polynomial $f$, a simple method for finding the roots to $f$ is to first construct its companion matrix $A$, then find the eigenvalues to $A$. The eigenvalues of $A$ are identical to the roots of $f$.
\end{remark}

\begin{theorem}\label{thm:w9_gen_fin_field}
    Given an irreducible polynomial $f$ over $\mathrm{GF}(p)$, let the companion matrix of $f$ be denoted by $A$, then the set
    \begin{equation}
        \mathcal{F} \defeq \setdef{b_0 + b_1 A + \cdots + b_mA^m}{b_i\in\mathrm{GF}(p), m>0} \subseteq \mathrm{GF}(p)^{d\times d}
    \end{equation}
    is a finite field of order $p^d$.
\end{theorem}
\begin{proof}
    First of all, again by Cayley--Hamilton theorem, $A^d$ can be represented as a degree $d-1$ polynomial of $A$, hence, we can restrict $m\le d-1$. The set $\mathcal{F}$ is thus finite. Next, we need to check the closure of the four operations.
    \begin{itemize}
        \item Both addition and subtraction are trivial.
        \item Multiplication is just the convolution of coefficients, with suitable application of the Cayley--Hamilton theorem from time to time.
        \item Division is more involved: Let us first consider how division works in $\mathrm{GF}(p)$. For the first few integers, division (finding their inverse) is easy, we have
        \begin{align*}
            \frac{1}{2} &= \frac{p+1}{2}, \\
            \frac{1}{3} &= \frac{p+1}{3} \text{ or } \frac{2p+1}{3}, \\
            \frac{1}{4} &= \left(\frac{1}{4}\right)^2, \cdots.
        \end{align*}
        However, in implementations, we can not afford this trial-and-error strategy of finding inverses, it is way too time-consuming. What we can do instead is applying a famous result in number theory -- the B\'ezout's lemma.
        \begin{lemma}[B\'ezout's Lemma]
            If $\mathrm{gcd}(p,k)=1$, then there exists integers $a$ and $b$ such that $ap+bk=1$.
        \end{lemma}
        This lemma can be proven by induction on the Euclidean algorithm.

        Then the procedure to finding $1/k$ for any $k\in\mathrm{GF}(p)$ will be to first find $ak+bp=1$ via the extended Euclidean algorithm, then since $ak\equiv1$ modulo $p$, we have $a=1/k$ in $\mathrm{GF}(p)$.

        Let us now extend the above to the set $\mathcal{F}$ and polynomials. For $f$ the characteristic polynomial of $A$, consider $f$ and $g$ are coprime polynomials in $\mathrm{GF}(p)[x]$. Coprimality of polynomials means that they do not share any factors, and since $f$ is irreducible, any $g$ over $\mathrm{GF}(p)$ must be coprime to $f$. By an extension to the B\'ezout's lemma, there exists polynomials $a$ and $b$ such that $af+bg=1$, the constant 1 polynomial. And since $f$ satisfies $f(A)=0$, we have that
        \begin{equation*}
            b(A) \cdot g(A) = \mathbbm{1},
        \end{equation*}
        meaning that $g(A)^{-1} = b(A)$.
    \end{itemize}
    Having all the operations checked, we can now safely state that the set $\mathcal{F}$ is indeed a field of order $\abs{F} = p^d$.
\end{proof}
\begin{example}
    We have the following field of
    \begin{equation*}
        \mathrm{GF}(4) \cong \Bigg\{ \underbrace{\left[\begin{matrix}
            0 & 0 \\ 0 & 0
        \end{matrix}\right]}_{0}, \underbrace{\left[\begin{matrix}
            1 & 0 \\ 0 & 1
        \end{matrix}\right]}_{1}, \underbrace{\left[\begin{matrix}
            1 & 1 \\ 1 & 0
        \end{matrix}\right]}_{A}, \underbrace{\left[\begin{matrix}
            0 & 1 \\ 1 & 1
        \end{matrix}\right]}_{B} \Bigg\}.
    \end{equation*}
    The set satisfies $1+A=B$, $1+B=A$, $AB=1$, $A^2=B$, and $B^2=A$. In fact, we can see that this field is generated by the degree 2 irreducible polynomial $f(x)=1+x+x^2$ over $\mathrm{GF}(2)$, where $B$ is the companion matrix of $f$.
\end{example}

In conclusion, we can generate the finite field $\mathrm{GF}(p^n)$ by considering an irreducible polynomial $f$ of degree $n-1$ over $\mathrm{GF}(p)$, then we find the companion matrix $A$ of $f$, lastly, we have the field extension $\mathrm{GF}(p^n) \cong \mathrm{GF}(p)[A]$.

\begin{remark}
    A more often-seen scheme to introducing finite fields of general orders is by the notion of a quotient. Given an irreducible polynomial $f\in\mathrm{GF}(p)[x]$, the quotient
    \begin{equation}
        \mathrm{GF}(p)[x] / (f(x))
    \end{equation}
    is a field.
\end{remark}

There are a few things we have not yet checked, however. To be precise, we have yet check the \textit{existence} of irreducible polynomials of any degree over $\mathrm{GF}(p)$, nor the \textit{uniqueness} of $\mathrm{GF}(p^n)$ when generated using different irreducible polynomials of same degree.

\subsection{Uniqueness of a Finite Field}
In this subsection, we will discuss the uniqueness of a finite field given its order. We further assume the existence of a finite field. Let $\mathbb{F}$ be a finite field, denote the abelian (commutative) group of
\begin{equation}
    \mathbb{F}^\times \defeq \setdef{B\in\mathbb{F}}{B^{-1}\text{ exists}} = \mathbb{F}\backslash \{0\}.
\end{equation}
We further denote $\abs{\mathbb{F}}=p^n$, and hence $\abs{\mathbb{F}^\times}=p^n-1$.

\begin{lemma}
    The set $\mathbb{F}^\times$ is a cyclic group, i.e., there exists an element $B\in\mathbb{F}$ such that
    \begin{equation*}
        \mathbb{F}^\times=\{1,B,B^2,\ldots,B^{p^{n}-2}\}.
    \end{equation*}
\end{lemma}
Before we proof this lemma, we must first show that it is indeed non-trivial. The more educated ones may argue that a finite abelian group can always be written as a finite direct product of cyclic groups: $G=\mathbb{Z}_{r_1}\times\mathbb{Z}_{r_2}\times\cdots$. So how can we be so sure that $\mathbb{F}^\times$ is a cyclic group and not a direct product of cyclic groups? A straightforward counter example will be that $\mathbb{F}\times\mathbb{F}$ is not a field! Since $(1,0)\cdot(0,1)=(0,0)$, $(1,0)$ is a zero divisor.
\begin{proof}
    For every $k$ in 1 to $p^n-1$, let us denote the set
    \begin{equation*}
        T_k = \setdef{B\in\mathbb{F}}{\mathrm{order}(B)=k},
    \end{equation*}
    where order of an element is defined by the minimum $k>0$ such that $B^k=1$. Define $S_k$ as the set of solutions to $x^k-1=0$, and $T_k\subseteq S_k$. Since the size of $S_k$ is at most $k$, $\abs{T_k} \le \abs{S_k} \le k$. From this, one can deduce that the number of elements in $T_k$ must be $\abs{T_k} \le \varphi(k)$. The function $\varphi(k)$ is the Euler totient function, denoting the amount of numbers coprime to $k$ from $1$ to $k$. Thus, we have
    \begin{equation*}
        \mathbb{F}^\times = \bigsqcup_{k\mid p^n-1}T_k \Rightarrow p^n-1=\abs{\mathbb{F}^\times} = \sum_{k\mid p^n-1} \abs{T_k} \le \sum_{k\mid p^n-1}\varphi(k) = p^n-1.
    \end{equation*}
    The last equality is a famous result from number theory. Hence, $\abs{T_k}=\varphi(k)$ for all $k\mid n$. The notation $k\mid n$ reads ``$n$ is divisible by $k$.'' Finally, we know there exists at least an element in $\mathbb{F}^\times$ that has order $p^n-1$, it is a cyclic group.
\end{proof}

Every $0\neq B\in\mathbb{F}$ satisfying $B^{p^n-1}=1$ is a root to $x^{p^n-1}-1=0$, there can be at most $p^n-1$ of them, and we have just shown that there are indeed $p^n-1$ of them. We often write
\begin{equation}
    x^{p^n-1} - 1= \prod_{B\in\mathbb{F},B\neq0}(x-B) \Longleftrightarrow x^{p^n} - x= \prod_{B\in\mathbb{F}}(x-B).
\end{equation}
Since all elements of a finite field of order $p^n$ must be the solution to the polynomial $x^{p^n}-x$, all finite fields of order $p^n$ are isomorphic to each other, i.e., their differences can only be a similarity transformation on the matrix elements.


\subsection{Existence of a Finite Field}
Lastly, let us prove the existence of a finite field by showing the existence of an irreducible polynomial of any order $d$ over any prime field $\mathrm{GF}(p)$.

We will show the existence by building from a couple of lemmas.
\begin{lemma}
    If $f$ is an irreducible polynomial of degree $d$ over $\mathrm{GF}(p)$, then $f\mid x^{p^d}-x$.
\end{lemma}
\begin{proof}
    Since $f(x)=\prod(x-B)$, where $B\in \mathrm{GF}(p)[x]/(f)\cong\mathrm{GF}(p^d)$ enumerates over all roots of $f$. And we also have $(x-B)\mid x^{p^d}-x$.
\end{proof}
\begin{lemma}
    For all $m$, $x^{p^d}-x\mid x^{p^{md}}-x$.
\end{lemma}
\begin{proof}
    Consider any root $B$ to $x^{p^d}-x=0$, it satisfies $B^{p^d}=B$. Then, $B^{p^{md}} = (B^{p^d})^{p^{(m-1)d}} = B^{p^{(m-1)d}} = \ldots = B$.
\end{proof}

Combining the above two lemmas, we obtain the following.
\begin{lemma}
    If $f$ is an irreducible polynomial of degree $d$, then $f\mid x^{p^{n}}-x$ where $d\mid n$.
\end{lemma}

This leads to a final step in showing the existence of $\mathrm{GF}(p^n)$:
\begin{theorem}
    Let $M(d)$ denote the number of irreducible polynomials over $\mathrm{GF}(p)$ of order $d$, then
    \begin{equation}
        \sum_{d\mid n} d \cdot M(d) = p^n
    \end{equation}
\end{theorem}
\begin{proof}
    Since for any irreducible polynomial $f$ over $\mathrm{GF}(p)$ of degree $d$ with, we have $f\mid x^{p^d}-x$ where $d\mid n$, then
    \begin{equation}
        \left.\prod_{f:\text{irred.}, \deg f\mid n} f \;\right\vert\; x^{p^n}-x.
    \end{equation}
    Counting the highest degree, we have
    \begin{equation*}
        \sum_{d\mid n} d \cdot M(d) \le p^n.
    \end{equation*}
    
    Next, for any $B\in\mathrm{GF}(p^n)$, let $f$ be its \textit{minimum polynomial}, which is the polynomial over $\mathrm{GF}(p)$ with minimum degree such that $B$ is root of it. Consider the field extension $\mathrm{GF}(p)(B)\cong\mathrm{GF}(p)[x]/(f)$. Since $B\in\mathrm{GF}(p^n)$, we have
    \begin{equation*}
        \underbrace{[\mathrm{GF}(p)(B):\mathrm{GF}(p)]}_{\deg f} \,\Big|\, [\mathrm{GF}(p^n):\mathrm{GF}(p)(B)].
    \end{equation*}
    The notation $[K:F]$ means the dimension of the vector space $K$ over the field $F$. Since each element in $B\in\mathrm{GF}(p^n)$ maps to an irreducible polynomial $f_B$ of degree $d\mid n$, and $f_B$ is mapped to at least $d$ times since it has $d$ distinct roots,
    \begin{align*}
        \abs{\mathrm{GF}(p^n)} = \#\setdef{B}{B\in\mathrm{GF}(p^n)} &= \sum_{d\mid n} \#\setdef{B}{\deg f_B=d} \\
        &\le \sum_{d\mid n} d \#\setdef{f\in\mathrm{GF}(p)[x]}{\exists\,B\in\mathrm{GF}(p^n), f_B = f, \deg f_B = d} \\
        &\le \sum_{d\mid n} d\cdot M(d).
    \end{align*}
    The equality is proven.
\end{proof}

Finally, since $M(d)\le p^d$, if $M(n)=0$, then
\begin{equation*}
    \sum_{d\mid n, d\neq n} d\cdot M(d) \le \sum_{d\mid n, d\neq n} d\cdot p^d \le p^{n/2} \sum_{d=1}^{n/2}d < p^n.
\end{equation*}
Hence, we can argue that $M(n)>0$, and that their exists a degree $n$ irreducible polynomial. And hence $\mathrm{GF}(p^n)$ exists.

A more elegant description of $M(n)>0$ is via the M\"obius transform:
\begin{theorem} \label{thm:w9_mobius}
    Let $\mu(d)\in\{+1,-1\}$ be the M\"obius function, we have
    \begin{align}
        M(n) &= \frac{1}{n} \sum_{d\mid n} \mu(d) \cdot p^{n/d}, \\
        p^{n} &= \sum_{d\mid n} d \cdot M(d).
    \end{align}
\end{theorem}

For another application of the M\"obius transform, one can refer to \aref{app:Greene}.

\begin{remark}
    From the \autoref{thm:w9_mobius} above, we see that
    \begin{equation}
        M(n) \sim \frac{1}{n} p^n \sim \frac{\#\text{ of polynomial up to degree }n}{\ln(\text{numerator})}.
    \end{equation}
    This is exactly the form of the \textit{prime number theorem} over function fields!

    The prime number theorem over the field $\mathbb{N}$ states the following: let $\pi(n)$ be the number of primes up to $n$, then
    \begin{equation}
        \pi(n) \sim \frac{N}{\ln N}.
    \end{equation}
    In fact, the prime number theorem holds for numbers, extension fields, function fields, and even permutations!
\end{remark}

\begin{remark}
    For all possible $p$ and $d$, we have shown that there exists an irreducible polynomial, however, which one is the best? I.e., how can we fix a representation that is ``good'' enough?

    The \textit{Conway polynomials} are the first such representation, standardizing the choosing of the irreducible polynomials via lexicographic ordering.
\end{remark}



\section{Reed--Solomon Codes over $\mathrm{GF}(p^n)$}
Let us upgrade from the simpler Reed--Solomon code introduced back in \autoref{sec:w7_RM_RS}.

\begin{definition}
    Given a prime number $p$ and an integer $n$. Choose an integer $k$ and another integer $N \le p^n$, we can select distinct \textit{evaluation points} $a_1$, $a_2$, $\ldots$, $a_N\in\mathrm{GF}(p^n)$. Then, the Reed--Solomon code is defined as
    \begin{equation}
        \mathrm{RS}(a_{1:N},k) \defeq \setdef{\left(f(a_1),f(a_2),\ldots,f(a_N)\right)}{f\in\mathrm{GF}(p^n)[x]^{<k}}
    \end{equation}
    One often writes $q$ in place of $p^n$.
\end{definition}

\begin{remark}
    The Reed--Solomon code is a really widely used code, appearing in our daily lives. For example the case where $p=2$ and $n=8$ is used in AES and QR codes. However, the polynomials used in the representation of AES are not the Conway polynomials, hence AES is often criticized for this reduce in efficiency.
\end{remark}

\begin{theorem}
    $\mathrm{RS}(a_{1:N},k)$ is an $[N,k,N-k+1]$-code over $\mathrm{GF}(q)$.
\end{theorem}
The proof is same as that of \autoref{thm:w7_RS_min_dist} by using an argument on the degrees of freedom.
\begin{proof}
    An equivalent statement is that for all polynomials $f,g\in\mathrm{GF}(q)[x]^{<k}$, $d=n-k+1$ of the evaluation points, say $a_{i_1}$ to $a_{i_d}$, satisfy
    \begin{equation*}
        f(a_{i_j})\neq g(a_{i_j})\;\;\;\;\;\forall\,j=1\sim d.
    \end{equation*}
    If the minimum distance of the Reed-Solomon code is less than $d$, say it is $N-k$, then any two polynomials should differ at $n-k$ points, and coincide at $k$ points. Without loss of generality, let $f$ and $g$ coincide at $a_1, a_2,\ldots,a_k$, then
    \begin{equation*}
        \underbrace{(x-a_1)(x-a_2)\cdots(x-a_k)}_{\text{degree }k}\mid \underbrace{f-g}_{\text{degree }<k}.
    \end{equation*}
    The only possibility is if $f-g=0$, which is a contradiction. Hence it is proven.
\end{proof}
\begin{lemma}[Singleton Bound]
    All $[N,k]$ linear codes have a minimum distance $\le N-k+1$.
\end{lemma}
Thus, we can see that Reed--Solomon codes are optimal in terms of their minimum distances by using the Singleton bound!

Next, we can also define the general Reed--Solomon code over $\mathrm{GF}(q)$ as
\begin{definition}
    Continue on the setting of a Reed--Solomon code. Select $c_1$, $c_2$, $\ldots$, $c_N\in\mathrm{GF}(q)$ that need not be distinct. The generalized Reed--Solomon code is defined as
    \begin{equation}
        \mathrm{GRS}(c_{1:N},a_{1:N},k) \defeq \setdef{\left(c_1f(a_1),c_2f(a_2),\ldots,c_Nf(a_N)\right)}{f\in\mathrm{GF}(q)[x]^{<k}}.
    \end{equation}
\end{definition}
When using the generalized Reed--Solomon code, one simply divides the received codeword by $c_i$'s to obtain the Reed--Solomon code. The only interesting thing and reason that we introduce it is that the dual of a Reed--Solomon code is $\mathrm{RS}^\perp = \mathrm{GRS}$.

\begin{example}
    A Reed--Solomon code is called a \textit{full Reed--Solomon code} when the length $N=q$. 
    \begin{theorem}[Dual of a Full Reed--Solomon Code]
        Pick any dimension $k$ and form the full Reed--Solomon code $\mathcal{C}$. Them, $\mathcal{C}^\perp$ is also a full Reed--Solomon code of dimension $q-k$.
    \end{theorem}
    \begin{proof}
        Consider any polynomial $f\in\mathrm{GF}(q)[x]^{<k}$ and $g\in\mathrm{GF}(q)[x]^{<q-k}$. Then
        \begin{equation*}
            \left(f(a_1),\ldots,f(a_q)\right) \cdot \left(g(a_1),\ldots,g(a_q)\right) = \sum_{i=1}^q (\underbrace{f\cdot g}_{h})(a_i).
        \end{equation*}
        For any $h$ of degree less than $q$, we have by symmetry
        \begin{equation*}
            \sum_{i=1}^q h(a_i) = 0.
        \end{equation*}
        Hence we see that $f$ and $g$ are always orthogonal, and the theorem is proven.
    \end{proof}
    The dual to a full Reed--Solomon code is really simple.
\end{example}

\begin{example}
    For non-full Reed--Solomon codes, we can also have a simple description of its dual by viewing them as punctured full Reed--Solomon codes. This proof is different from the one given in \autoref{thm:w7_dual_GRS}. By the following diagram:
    \begin{center}
    \begin{tikzcd}
	{\mathcal{C}} && {\mathrm{Pun}_N(\mathcal{C})} \\
	{\mathcal{C}^\perp} && {\mathrm{Sho}_N(\mathcal{C}^\perp)}
	\arrow["{\mathrm{Pun}_N}", from=1-1, to=1-3]
	\arrow["\perp"', from=1-1, to=2-1]
	\arrow["\perp", from=1-3, to=2-3]
	\arrow["{\mathrm{Sho}_N}", from=2-1, to=2-3]
    \end{tikzcd}
    \end{center}
    We have, for the code
    \begin{equation*}
        \mathrm{Pun}_N(\mathcal{C}) = \setdef{\left(f(a_1),\ldots,f(a_{N-1})\right)}{f\in\mathrm{GF}(q)[x]^{<k}},
    \end{equation*}
    its dual code being
    \begin{align*}
        \mathrm{Sho}_N(\mathcal{C}^\perp) &= \mathrm{Sho}_N \setdef{\left(g(a_1),\ldots,g(a_{N-1}),g(a_N)\right)}{g\in\mathrm{GF}(q)[x]^{<N-k}} \\
        &= \setdef{\left(g(a_1),\ldots,g(a_{N-1})\right)}{g\in\mathrm{GF}(q)[x]^{<N-k}, g(a_N)=0} \\
        &= \setdef{\left(g(a_1),\ldots,g(a_{N-1})\right)}{g(x)=h(x)\cdot(x-a_N), h\in\mathrm{GF}(q)[x]^{< N-k-1}} \\
        &= \setdef{\left((a_1-a_N)h(a_1),\ldots,(a_{N-1}-a_N)h(a_{N-1})\right)}{h\in\mathrm{GF}(q)[x]^{< N-k-1}}.
    \end{align*}
    By continuing puncturing the code, we will have more Vandermonde-like terms in the front, and finally obtaining the dual code to a general Reed--Solomon code as in \autoref{thm:w7_dual_GRS}.
\end{example}