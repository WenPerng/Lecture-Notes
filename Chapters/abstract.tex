This is the course on Modern Coding Theory and Technologies given by professor Hsin-Po Wang in the spring semester of 2025.

The lecture goes over the modern coding theory on polar codes and LDPC codes. Many of the properties and the required mathematical preliminaries such as concentration inequalities, martingale theory, and field theory are introduced along the way.

Besides on coding theory, professor Wang also has deep interest in group testing and matrix multiplication. Many such detours on diverse topics are given to broaden the students' horizon.

\vspace{1cm}
\hrule

Here is a brief description of the items covered in each week:
\begin{itemize}
    \item \textbf{Week 1 (2/18)}: an introduction to modern coding theory, covering the origin of polar coding and binary erasure channel.
    \item \textbf{Week 2 (2/25)}: basic knowledge on concentration bounds and Shannon's channel coding theorem, the wiring structure to polar code is written down.
    \item \textbf{Week 3 (3/04)}: special attention on decoding procedure, introduced martingale for applying on polar code over BEC, decomposition of BSC from analyzing BMSC.
    \item \textbf{Week 4 (3/11)}: large deviation theory and its relation with martingale theory, Reed-Muller code. Further discussion on source coding through Shannon's method, arithmetic encoding, and source polarization arguments.
    \item \textbf{Week 5 (3/18)}: channel parameters + martingale theory to give more characterization of polar transformation. Distributed source coding (Slepian-Wolf).
    \item \textbf{Week 6 (03/25)}: algebra of linear code, the functional equation regarding the speed of polarization, showing how polar code can do both source and channel coding.
    \item \textbf{Week 7 (04/01)}: a mock test on the midterm, covering many of the mentioned unproven results from previous lectures.
    \item \textbf{Week 8 (04/08)}: midterm.
    \item \textbf{Week 9 (04/15)}: the construction, existence and uniqueness of finite fields. Further applying the result to Reed--Solomon codes.
    \item \textbf{Week 10 (04/22)}: introduction to LDPC, also the evolution of erasure of LDPC over BEC. A detour into entropy complexity.
    \item \textbf{Week 11 (04/29)}: how to decode RS code, the quality evolution of LDPC code over BSC, solving group testing using RS code.
    \item \textbf{Week 12 (05/06)}: More on group testing! The binary search method and the subsequent discussions on progeny are also interesting.
    \item \textbf{Week 13} (05/13): Tradeoffs between different aspects of group testing. Distributed matrix multiplication.
    \item \textbf{Week 14}:
    \item \textbf{Week 15}:
    \item \textbf{Week 16}:
\end{itemize}
% For a more detailed description and summary of what is taught every week, please refer to \autoref{sec:weekly_summary}.

\vspace{1cm}
\hrule
To be updated (2025/05/18):
\begin{itemize}
    \item Section 4.7 on the application of RS code on group testing.
    \item Appendix on Catalan numbers.
    \item W12 and W13 on group testing.
    \item References on algorithms for group testing.
    \item W13 on fast matrix multiplication.
\end{itemize}